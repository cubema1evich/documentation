\section{Техническое задание}
\subsection{Основание для разработки}

Основанием для разработки является задание на выпускную квалификационную работу бакалавра "<Бизнес-проект: Закрытый корпоративный мессенджер">.

\subsection{Цель и назначение разработки}

Разработка защищенного корпоративного мессенджера для внутреннего обмена сообщениями сотрудников компании с функциями:
\begin{itemize}
	\item Шифрование передаваемых сообщений
	\item Групповое взаимодействие с ролевым доступом
	\item Хранение истории переписки
\end{itemize}


\subsection{Функционал мессенджера}

\subsubsection{Аутентификация}
\begin{itemize}
	\item Регистрация по корпоративному логину
	\item Авторизация с проверкой учетных данных
\end{itemize}

\subsubsection{Чаты и сообщения}
\begin{itemize}
	\item Личные сообщения между сотрудниками
	\item Групповые чаты с управлением участниками
	\item Отправка текстовых сообщений и файлов
	\item Поиск по истории сообщений
\end{itemize}

\subsubsection{Управление группами}
\begin{itemize}
	\item Создание/удаление чатов
	\item Добавление/исключение участников
	\item Назначение ролей (владелец, администратор участник)
\end{itemize}

\subsection{Роли пользователей}

\begin{xltabular}{\textwidth}{|l|X|}
	\caption{Роли пользователей}\label{tab:roles} \\ \hline
	\centrow Роль & \centrow Права \\ \hline
	\endfirsthead
	Обычный пользователь & 
	\begin{itemize}
		\item Личная переписка
		\item Участие в групповых чатах
		\item Отправка сообщений и файлов
	\end{itemize} \\ \hline
	Администратор чата & 
	\begin{itemize}
		\item Все права обычного пользователя
		\item Добавление/удаление участников
		\item Переименование чата
	\end{itemize} \\ \hline
	Владелец &
	\begin{itemize}
		Все права доступа
	\end{itemize} \\ \hline
\end{xltabular}

\subsubsection {Тестирование и отладка:}
\begin{itemize}
	\item Проверка безопасности
	\item Функциональное тестирование интерфейсов
	\item Тестирование удобства использования
\end{itemize}
\newpage
\subsection{Требования к интерфейсу}

\subsubsection{Основные элементы интерфейса}

\begin{figure}[ht]
	\centering
	\includegraphics[width=0.8\linewidth]{"images/UI макет"}
	\caption{Схема интерфейса мессенджера}
	\label{fig:ui-main}
\end{figure}

\begin{figure}[ht]
	\centering
	\includegraphics[width=0.8\linewidth]{"images/UI макет регистрации"}
	\caption{Схема интерфейса формы регистрации}
	\label{fig:ui-reg}
\end{figure}

\begin{figure}[ht]
	\centering
	\includegraphics[width=0.8\linewidth]{"images/UI макет авторизации"}
	\caption{Схема интерфейса формы авторизации}
	\label{fig:ui-auth}
\end{figure}

\subsection{Сценарии использования}

\subsubsection{Регистрация нового пользователя}
\begin{enumerate}
	\item Пользователь нажимает кнопку "Зарегистрироваться" на форме авторизации
	\item Система отображает форму регистрации (рис. \ref{fig:ui-reg}) с полями:
	\begin{itemize}
		\item Корпоративная почта (логин)
		\item Пароль (с требованиями сложности)
		\item Подтверждение пароля
	\end{itemize}
	\item Пользователь заполняет все обязательные поля
	\item Пользователь нажимает кнопку "Зарегистрироваться"
	\item Система проверяет данные:
	\begin{itemize}
		\item При корректных данных:
		\begin{itemize}
			\item Создает новую учетную запись
			\item Отправляет подтверждение на корпоративную почту
			\item Перенаправляет на форму авторизации
			\item Выводит сообщение "Регистрация успешно завершена"
		\end{itemize}
		\item При ошибках:
		\begin{itemize}
			\item Выделяет проблемные поля
			\item Показывает соответствующие сообщения об ошибках:
			\begin{enumerate}
				\item "Пароль должен содержать не менее 8 символов"
				\item "Пароли не совпадают"
				\item "Учетная запись с таким именем уже существует"
			\end{enumerate}
		\end{itemize}
	\end{itemize}
	\item После успешной регистрации администратор получает уведомление о новом пользователе для подтверждения корпоративного доступа
\end{enumerate}

\subsubsection{Авторизация пользователя}
\begin{enumerate}
	\item Пользователь открывает веб-интерфейс мессенджера
	\item Система отображает форму авторизации (рис. \ref{fig:ui-auth})
	\item Пользователь вводит корпоративный логин и пароль
	\item Пользователь нажимает кнопку "Войти"
	\item Система проверяет учетные данные:
	\begin{itemize}
		\item При успехе - загружает основной интерфейс (рис. \ref{fig:ui-main})
		\item При ошибке - показывает сообщение "Неверный логин или пароль"
	\end{itemize}
	\item При нажатии "Зарегистрироваться" система перенаправляет на форму регистрации (рис. \ref{fig:ui-reg})
\end{enumerate}

\subsubsection{Создание группового чата}
\begin{enumerate}
	\item Пользователь нажимает кнопку "+" (Создать чат) на левой панели
	\item Система отображает диалоговое окно:
	\begin{itemize}
		\item Поле ввода названия чата
		\item Список доступных сотрудников
		\item Чекбоксы для выбора участников
	\end{itemize}
	\item Пользователь вводит название чата
	\item Пользователь отмечает нужных участников
	\item Пользователь нажимает кнопку "Создать"
	\item Система:
	\begin{itemize}
		\item Создает новый чат
		\item Добавляет выбранных участников
		\item Отображает новый чат в списке
	\end{itemize}
\end{enumerate}

\subsubsection{Отправка сообщений}
\begin{enumerate}
	\item Пользователь выбирает чат из списка
	\item Система загружает историю переписки
	\item Пользователь вводит текст в нижнее поле ввода
	\item Пользователь может:
	\begin{itemize}
		\item Нажать кнопку "Отправить" (или Enter)
		\item Нажать кнопку "Прикрепить файл" и выбрать файл
	\end{itemize}
	\item Система:
	\begin{itemize}
		\item Шифрует и отправляет сообщение
		\item Отображает сообщение в истории чата
		\item Для файлов - показывает превью и название
	\end{itemize}
\end{enumerate}

\subsubsection{Управление участниками группы (для администраторов)}
\begin{enumerate}
	\item Пользователь открывает групповой чат
	\item Пользователь нажимает иконку "Управление чатом" в заголовке
	\item Система отображает меню:
	\begin{itemize}
		\item "Добавить участника"
		\item "Исключить участника"
		\item "Назначить администратора"
	\end{itemize}
	\item При выборе "Добавить участника":
	\begin{itemize}
		\item Открывается список сотрудников
		\item Администратор выбирает сотрудников
		\item Нажимает "Добавить"
		\item Система присылает уведомление новым участникам
	\end{itemize}
	\item При выборе "Исключить участника":
	\begin{itemize}
		\item Открывается список текущих участников
		\item Администратор выбирает участника
		\item Нажимает "Исключить"
		\item Система удаляет участника из чата
	\end{itemize}
\end{enumerate}

\subsection{Требования к оформлению документации}

Документация должна соответствовать ГОСТ 19.102-77 и ГОСТ 34.601-90. Единая система программной документации.