\section{Технический проект}
\subsection{Общая характеристика организации решения задачи}

Разрабатывается серверная часть веб-приложения для чат-платформы, обеспечивающая REST API для взаимодействия между клиентской частью и базой данных. Сервер реализует следующие основные функции:
\begin{itemize}
	\item Аутентификация и авторизация пользователей
	\item Обмен сообщениями в групповых чатах
	\item Личная переписка между пользователями
	\item Управление группами и правами участников
	\item Работа с вложениями и медиафайлами
\end{itemize}

\subsection{Обоснование выбора технологии проектирования}

Для реализации серверной части выбраны следующие технологии:

\subsubsection{Язык программирования Python}

Python выбран благодаря:
\begin{itemize}
	\item Простоте и читаемости кода
	\item Богатой экосистеме веб-фреймворков
	\item Хорошей поддержке работы с базами данных
	\item Кроссплатформенности
\end{itemize}

\subsubsection{Фреймворк WSGI}

В качестве основы сервера используется WSGI (Web Server Gateway Interface) - стандартный интерфейс между веб-сервером и Python-приложениями. Преимущества:
\begin{itemize}
	\item Легковесность
	\item Простота развертывания
	\item Совместимость с различными серверами (Waitress, Gunicorn и др.)
\end{itemize}

\subsubsection{База данных SQLite}

SQLite выбрана как:
\begin{itemize}
	\item Встроенное решение, не требующее отдельного сервера
	\item Простое в настройке и использовании
\end{itemize}

\subsection{Диаграмма компонентов}

На рисунке \ref{fig:-components} представлена архитектура серверной части приложения.

\begin{figure}[ht]
	\centering
	\includegraphics[width=0.8\linewidth]{"images/Диаграмма компонентов"}
	\caption{Диаграмма компонентов серверной части}
	\label{fig:-components}
\end{figure}

\subsection{Основные API-эндпоинты}

\subsubsection{Аутентификация и пользователи}

\begin{xltabular}{\textwidth}{|l|l|p{1.7cm}|X|}
	\caption{API для работы с пользователями}\label{tab:users_api} \\ \hline
	\centrow Поле & \centrow Тип & \centrow Метод & \centrow Описание \\ \hline
	\thead{1} & \thead{2} & \centrow 3 & \centrow 4 \\ \hline
	\endfirsthead
	\continuecaption{Продолжение таблицы \ref{tab:users_api}}
	\thead{1} & \thead{2} & \centrow 3 & \centrow 4 \\ \hline
	\finishhead
	/register & Регистрация & POST & Создание нового пользователя. Параметры: username, password \\ \hline 
	/login & Аутентификация & POST & Вход в систему. Параметры: username, password \\ \hline 
	/get\_user\_id & Информация & GET & Получение данных текущего пользователя \\ \hline 
\end{xltabular}

\subsubsection{Работа с сообщениями}

\begin{xltabular}{\textwidth}{|l|l|p{1.7cm}|X|}
	\caption{API для работы с сообщениями}\label{tab:messages_api} \\ \hline
	\centrow Поле & \centrow Тип & \centrow Метод & \centrow Описание \\ \hline
	\thead{1} & \thead{2} & \centrow 3 & \centrow 4 \\ \hline
	\endfirsthead
	\continuecaption{Продолжение таблицы \ref{tab:messages_api}}
	\thead{1} & \thead{2} & \centrow 3 & \centrow 4 \\ \hline
	\finishhead
	/get\_messages & Получение & GET & Запрос новых сообщений. Параметры: timestamp \\ \hline 
	/send\_message & Отправка & POST & Отправка сообщения. Параметры: message, files[] \\ \hline 
	/delete\_message/\{id\} & Удаление & DELETE & Удаление сообщения по ID \\ \hline 
	/edit\_message/\{id\} & Редактирование & PUT & Изменение сообщения. Параметры: message \\ \hline 
	/check\_messages & Проверка & GET & Проверка существования сообщений. Параметры: ids[] \\ \hline 
	/check\_edited\_messages & Проверка & GET & Поиск измененных сообщений. Параметры: last\_timestamp \\ \hline 
\end{xltabular}
\newpage
\subsubsection{Групповые чаты}

\begin{xltabular}{\textwidth}{|l|l|p{1.7cm}|X|}
	\caption{API для работы с группами}\label{tab:groups_api} \\ \hline
	\centrow Поле & \centrow Тип & \centrow Метод & \centrow Описание \\ \hline
	\thead{1} & \thead{2} & \centrow 3 & \centrow 4 \\ \hline
	\endfirsthead
	\continuecaption{Продолжение таблицы \ref{tab:groups_api}}
	\thead{1} & \thead{2} & \centrow 3 & \centrow 4 \\ \hline
	\finishhead
	/create\_group & Создание & POST & Создание новой группы. Параметры: name \\ \hline 
	/add\_to\_group & Добавление & POST & Приглашение пользователя. Параметры: group\_id, username \\ \hline 
	/get\_groups & Список & GET & Получение групп пользователя \\ \hline 
	/get\_group\_messages & Сообщения & GET & Получение сообщений группы. Параметры: group\_id \\ \hline 
	/get\_group\_members & Участники & GET & Получение списка участников. Параметры: group\_id \\ \hline 
	/leave\_group & Выход & POST & Покидание группы. Параметры: group\_id \\ \hline 
	/change\_member\_role & Права & POST & Изменение роли. Параметры: group\_id, username, role \\ \hline 
	/rename\_group & Переименование & POST & Изменение названия. Параметры: group\_id, new\_name \\ \hline 
	/remove\_from\_group & Исключение & POST & Удаление участника. Параметры: group\_id, username \\ \hline 
\end{xltabular}

\subsubsection{Личные сообщения}

\begin{xltabular}{\textwidth}{|l|l|p{1.7cm}|X|}
	\caption{API для личных сообщений}\label{tab:private_api} \\ \hline
	\centrow Поле & \centrow Тип & \centrow Метод & \centrow Описание \\ \hline
	\thead{1} & \thead{2} & \centrow 3 & \centrow 4 \\ \hline
	\endfirsthead
	\continuecaption{Продолжение таблицы \ref{tab:private_api}}
	\thead{1} & \thead{2} & \centrow 3 & \centrow 4 \\ \hline
	\finishhead
	/send\_private\_message & Отправка & POST & Отправка ЛС. Параметры: receiver, message \\ \hline 
	/get\_private\_messages & Получение & GET & Получение ЛС. Параметры: user, timestamp \\ \hline 
	/get\_private\_chats & Список & GET & Получение чатов пользователя \\ \hline 
	/search\_users & Поиск & GET & Поиск пользователей. Параметры: q \\ \hline 
\end{xltabular}

\subsection{Основные классы и функции}

\subsubsection{Базовый класс View}

Базовый класс для всех представлений, реализует:
\begin{itemize}
	\item Обработку HTTP-запросов
	\item Чтение файлов
	\item Формирование ответов
\end{itemize}

\subsubsection{Классы представлений}

Основные классы представлений:
\begin{itemize}
	\item \textbf{TemplateView} - базовый класс для шаблонов
	\item \textbf{IndexView} - главная страница
	\item \textbf{GetMessageView} - получение сообщений
	\item \textbf{SendMessageView} - отправка сообщений
	\item \textbf{RegisterView/LoginView} - аутентификация
	\item \textbf{Group-related views} - управление группами
	\item \textbf{PrivateMessage views} - личные сообщения
\end{itemize}

\subsubsection{Вспомогательные функции}

\begin{itemize}
	\item \textbf{json\_response} - формирование JSON-ответа
	\item \textbf{forbidden\_response} - ответ 403 Forbidden
	\item \textbf{get\_mime} - определение MIME-типа файла
\end{itemize}

\subsection{Схема базы данных}

Основные таблицы базы данных:
\begin{itemize}
	\item \textbf{users} - информация о пользователях
	\item \textbf{group\_messages} - сообщения групп
	\item \textbf{private\_messages} - личные сообщения
	\item \textbf{groups} - информация о группах
	\item \textbf{group\_members} - участники групп
	\item \textbf{attachments} - вложения к сообщениям
\end{itemize}

\subsection{Безопасность}

Реализованные механизмы безопасности:
\begin{itemize}
	\item Хеширование паролей (bcrypt)
	\item Валидация входных данных
	\item Проверка прав доступа
\end{itemize}

\subsection{Обработка ошибок}

Реализованы специальные представления для обработки ошибок:
\begin{itemize}
	\item \textbf{NotFoundView} - 404 Not Found
	\item \textbf{ForbiddenView} - 403 Forbidden
	\item \textbf{InternalServerErrorView} - 500 Internal Server Error
\end{itemize}