\section{Технический проект}
\subsection{Общая характеристика организации решения задачи}

Разрабатывается серверная часть веб-приложения для чат-платформы, обеспечивающая API для взаимодействия между клиентской частью и базой данных. Сервер реализует следующие основные функции:
\begin{itemize}
	\item Аутентификация и авторизация пользователей
	\item Обмен сообщениями в групповых чатах
	\item Личная переписка между пользователями
	\item Управление группами и правами участников
	\item Работа с вложениями и медиафайлами
\end{itemize}

\subsection{Обоснование выбора технологии проектирования}

Для реализации серверной части выбраны следующие технологии:

\subsubsection{Язык программирования Python}

Python выбран благодаря:
\begin{itemize}
	\item Простоте и читаемости кода
	\item Богатой экосистеме веб-фреймворков
	\item Хорошей поддержке работы с базами данных
	\item Кроссплатформенности
\end{itemize}

\subsubsection{Фреймворк WSGI}

В качестве основы сервера используется WSGI (Web Server Gateway Interface) - стандартный интерфейс между веб-сервером и Python-приложениями. Преимущества:
\begin{itemize}
	\item Легковесность
	\item Простота развертывания
	\item Совместимость с различными серверами (Waitress, Gunicorn и др.)
\end{itemize}

\subsubsection{База данных SQLite}

SQLite выбрана как:
\begin{itemize}
	\item Встроенное решение, не требующее отдельного сервера
	\item Простое в настройке и использовании
\end{itemize}

\subsection{Диаграмма компонентов}

На рисунке \ref{fig:-components} представлена архитектура серверной части приложения.

\begin{figure}[ht]
	\centering
	\includegraphics[width=0.9\linewidth]{"images/Диаграмма компонентов"}
	\caption{Диаграмма компонентов серверной части}
	\label{fig:-components}
\end{figure}

\subsection{Основные API-эндпоинты}

\subsubsection{Аутентификация и пользователи}

\begin{xltabular}{\textwidth}{|l|l|p{1.7cm}|X|}
	\caption{API для работы с пользователями}\label{tab:users_api} \\ \hline
	\centrow Поле & \centrow Тип & \centrow Метод & \centrow Описание \\ \hline
	\thead{1} & \thead{2} & \centrow 3 & \centrow 4 \\ \hline
	\endfirsthead
	\continuecaption{Продолжение таблицы \ref{tab:users_api}}
	\thead{1} & \thead{2} & \centrow 3 & \centrow 4 \\ \hline
	\finishhead
	/register & Регистрация & POST & Создание нового пользователя. Параметры: username, password \\ \hline 
	/login & Аутентификация & POST & Вход в систему. Параметры: username, password \\ \hline 
	/get\_user\_id & Информация & GET & Получение данных текущего пользователя \\ \hline 
\end{xltabular}

\subsubsection{Работа с сообщениями}

\begin{xltabular}{\textwidth}{|l|l|p{1.7cm}|X|}
	\caption{API для работы с сообщениями}\label{tab:messages_api} \\ \hline
	\centrow Поле & \centrow Тип & \centrow Метод & \centrow Описание \\ \hline
	\thead{1} & \thead{2} & \centrow 3 & \centrow 4 \\ \hline
	\endfirsthead
	\continuecaption{Продолжение таблицы \ref{tab:messages_api}}
	\thead{1} & \thead{2} & \centrow 3 & \centrow 4 \\ \hline
	\finishhead
	/get\_messages & Получение & GET & Запрос новых сообщений. Параметры: timestamp \\ \hline 
	/send\_message & Отправка & POST & Отправка сообщения. Параметры: message, files[] \\ \hline 
	/delete\_message/\{id\} & Удаление & DELETE & Удаление сообщения по ID \\ \hline 
	/edit\_message/\{id\} & Редактирование & PUT & Изменение сообщения. Параметры: message \\ \hline 
	/check\_messages & Проверка & GET & Проверка существования сообщений. Параметры: ids[] \\ \hline 
	/check\_edited\_messages & Проверка & GET & Поиск измененных сообщений. Параметры: last\_timestamp \\ \hline 
\end{xltabular}
\newpage
\subsubsection{Групповые чаты}

\begin{xltabular}{\textwidth}{|l|l|p{1.7cm}|X|}
	\caption{API для работы с группами}\label{tab:groups_api} \\ \hline
	\centrow Поле & \centrow Тип & \centrow Метод & \centrow Описание \\ \hline
	\thead{1} & \thead{2} & \centrow 3 & \centrow 4 \\ \hline
	\endfirsthead
	\continuecaption{Продолжение таблицы \ref{tab:groups_api}}
	\thead{1} & \thead{2} & \centrow 3 & \centrow 4 \\ \hline
	\finishhead
	/create\_group & Создание & POST & Создание новой группы. Параметры: name \\ \hline 
	/add\_to\_group & Добавление & POST & Приглашение пользователя. Параметры: group\_id, username \\ \hline 
	/get\_groups & Список & GET & Получение групп пользователя \\ \hline 
	/get\_group\_messages & Сообщения & GET & Получение сообщений группы. Параметры: group\_id \\ \hline 
	/get\_group\_members & Участники & GET & Получение списка участников. Параметры: group\_id \\ \hline 
	/leave\_group & Выход & POST & Покидание группы. Параметры: group\_id \\ \hline 
	/change\_member\_role & Права & POST & Изменение роли. Параметры: group\_id, username, role \\ \hline 
	/rename\_group & Переименование & POST & Изменение названия. Параметры: group\_id, new\_name \\ \hline 
	/remove\_from\_group & Исключение & POST & Удаление участника. Параметры: group\_id, username \\ \hline 
\end{xltabular}

\subsubsection{Личные сообщения}

\begin{xltabular}{\textwidth}{|l|l|p{1.7cm}|X|}
	\caption{API для личных сообщений}\label{tab:private_api} \\ \hline
	\centrow Поле & \centrow Тип & \centrow Метод & \centrow Описание \\ \hline
	\thead{1} & \thead{2} & \centrow 3 & \centrow 4 \\ \hline
	\endfirsthead
	\continuecaption{Продолжение таблицы \ref{tab:private_api}}
	\thead{1} & \thead{2} & \centrow 3 & \centrow 4 \\ \hline
	\finishhead
	/send\_private\_message & Отправка & POST & Отправка ЛС. Параметры: receiver, message \\ \hline 
	/get\_private\_messages & Получение & GET & Получение ЛС. Параметры: user, timestamp \\ \hline 
	/get\_private\_chats & Список & GET & Получение чатов пользователя \\ \hline 
	/search\_users & Поиск & GET & Поиск пользователей. Параметры: q \\ \hline 
\end{xltabular}

\subsubsection{Дополнительные API}

\begin{xltabular}{\textwidth}{|l|l|p{1.7cm}|X|}
	\caption{Дополнительные API эндпоинты}\label{tab:additional_api} \\ \hline
	\centrow Поле & \centrow Тип & \centrow Метод & \centrow Описание \\ \hline
	\thead{1} & \thead{2} & \centrow 3 & \centrow 4 \\ \hline
	\endfirsthead
	\continuecaption{Продолжение таблицы \ref{tab:additional_api}}
	\thead{1} & \thead{2} & \centrow 3 & \centrow 4 \\ \hline
	\finishhead
	/get\_group\_name & Информация & GET & Получение названия группы по ID \\ \hline 
	/send\_system\_message & Системное & POST & Отправка системных сообщений \\ \hline 
	/check\_groups\_updates & Проверка & GET & Проверка обновлений в группах \\ \hline 
	/check\_private\_chats\_updates & Проверка & GET & Проверка новых ЛС \\ \hline 
\end{xltabular}


\subsection{Основные классы и функции}

Модульная организация кода классов представлений и вспомогательных функций. Классы и функции распределены по специализированным модулям внутри директории views (а также в отдельных модулях для утилит). Такой подход позволяет обеспечить более удобное чтение кода, модульное тестирование и масштабируемость системы.

\subsubsection{Базовый модуль base.py} 
Базовый модуль, реализующий общий функционал для всех представлений, включает:
\begin{itemize}
	\item \textbf{View} класс для обработки базовых HTTP-запросов и формирования стандартных ответов;
	\item \textbf{TemplateView} класс для работы с файловыми шаблонами и генерации HTML-страниц;
	\item \textbf{IndexView} класс для отображения главной страницы приложения.
\end{itemize}
Данный модуль отвечает за:
\begin{itemize}
	\item Обработку HTTP-запросов;
	\item Чтение и интерпретацию файлов шаблонов;
	\item Формирование ответов сервера.
\end{itemize}

\subsubsection{Модуль auth.py}
В данном модуле размещены классы, связанные с работой пользователей:
\begin{itemize}
	\item \textbf{RegisterView} класс для обработки регистрации новых пользователей;
	\item \textbf{LoginView} класс для обработки аутентификации и входа в систему.
\end{itemize}

\subsubsection{Модуль group.py}
В данном модуле размещены классы, связанные с работой групп:
\begin{itemize}
	\item \textbf{GetGroupMembersView} класс для получения списка участников группы;
	\item \textbf{LeaveGroupView} класс для выхода пользователя из группы;
	\item \textbf{CreateGroupView} класс для создания новой группы;
	\item \textbf{AddToGroupView} класс для добавления пользователей в группу;
	\item \textbf{GetGroupView} класс для получения подробной информации о группе;
	\item \textbf{CheckGroupAccessView} класс для проверки прав доступа пользователя в группе;
	\item \textbf{GetGroupNameView} класс для получения названия группы по её идентификатору;
	\item \textbf{CheckGroupsUpdateView} класс для проверки обновлений в информации о группах;
	\item \textbf{RenameGroupView} класс для переименования группы;
	\item \textbf{RemoveFromGroupView} класс для удаления участников из группы.
\end{itemize}
Данный модуль отвечает за:
\begin{itemize}
	\item Управление данными групп пользователей;
	\item Контроль доступа и администрирование групп;
	\item Обеспечение актуальности информации о группах.
\end{itemize}

\subsubsection{Модуль error.py}
В данном модуле размещены классы, связанные с обработкой ошибок:
\begin{itemize}
	\item \textbf{NotFoundView} класс для отображения ошибки 404 (Не найдено);
	\item \textbf{InternalServerErrorView} класс для обработки ошибки 500 (Внутренняя ошибка сервера);
	\item \textbf{ForbiddenView} класс для генерации ответа 403 (Доступ запрещён).
\end{itemize}
Данный модуль отвечает за:
\begin{itemize}
	\item Обработку исключительных ситуаций и ошибок;
	\item Вывод информативных сообщений при сбоях;
	\item Защиту приложения от некорректных запросов.
\end{itemize}

\subsubsection{Модуль message.py}
В данном модуле размещены классы, связанные с обработкой сообщений:
\begin{itemize}
	\item \textbf{GetMessageView} класс для получения списка сообщений;
	\item \textbf{SendMessageView} класс для отправки новых сообщений;
	\item \textbf{DeleteMessageView} класс для удаления сообщений по идентификатору;
	\item \textbf{EditMessageView} класс для редактирования ранее отправленных сообщений;
	\item \textbf{GetGroupMessageView} класс для получения сообщений группового чата;
	\item \textbf{GetPrivateMessageView} класс для получения личных сообщений;
	\item \textbf{SendPrivateMessageView} класс для отправки личных сообщений;
	\item \textbf{SendSystemMessageView} класс для отправки системных уведомлений;
	\item \textbf{CheckMessagesView} класс для проверки наличия новых сообщений;
	\item \textbf{CheckEditedMessageView} класс для проверки изменений в сообщениях.
\end{itemize}
Данный модуль отвечает за:
\begin{itemize}
	\item Организацию процесса обмена сообщениями между несколькими пользователями;
	\item Поддержание целостности переписок;
	\item Обеспечение возможности модификации и удаления сообщений.
\end{itemize}

\subsubsection{Модуль pchat.py}
В данном модуле размещены классы, связанные с обработкой личных чатов:
\begin{itemize}
	\item \textbf{GetPrivateChatsView} класс для получения списка личных чатов пользователя;
	\item \textbf{CheckPrivateChatsUpdateView} класс для проверки обновлений в личных чатах.
\end{itemize}
Данный модуль отвечает за:
\begin{itemize}
	\item Организацию и управление личными чатами;
	\item Обеспечение оперативного доступа к истории переписки.
\end{itemize}

\subsubsection{Модуль search.py}
В данном модуле размещены классы, связанные с поиском:
\begin{itemize}
	\item \textbf{SearchUsersView} класс для поиска пользователей по заданным критериям (например, по имени или email);
	\item \textbf{SearchMessagesView} класс для поиска сообщений по содержимому и дате.
\end{itemize}
Данный модуль отвечает за:
\begin{itemize}
	\item Обеспечение функционала поиска по базе данных пользователей и сообщений;
	\item Быстрое извлечение релевантных данных;
	\item Улучшение навигации пользователя по приложению.
\end{itemize}

\subsubsection{Модуль users.py}
В данном модуле размещены классы, связанные с обработкой данных пользователей:
\begin{itemize}
	\item \textbf{GetUserIDView} класс для получения идентификатора текущего пользователя;
	\item \textbf{GetGeneralMembersView} класс для получения списка всех пользователей системы;
	\item \textbf{ChangeMemberRoleView} класс для изменения роли участника в группе.
\end{itemize}
Данный модуль отвечает за:
\begin{itemize}
	\item Управление информацией о пользователях;
	\item Администрирование и обновление пользовательских данных;
	\item Поддержание актуальности и целостности данных системы.
\end{itemize}


\paragraph{Вспомогательные модули.}
Для поддержки работы основных классов выделены несколько вспомогательных модулей: \texttt{routes}, \texttt{mimes}, \texttt{utils}, \texttt{app} и \texttt{run}, каждый из которых реализует специфичный набор функций:
\begin{itemize}
	\item \textbf{routes} обеспечивает маршрутизацию HTTP-запросов к соответствующим представлениям, что позволяет определить, какой модуль и функция должны обработать тот или иной запрос;
	\item \textbf{mimes} реализует функции для определения MIME-типов файлов на основе расширений или содержимого, обеспечивая корректное формирование заголовков ответов;
	\item \textbf{utils} содержит набор вспомогательных утилит, включая функции для генерации JSON-ответов, обработки ошибок, а также хеширования и проверки паролей;
	\item \textbf{app} отвечает за инициализацию и конфигурацию серверного приложения, настройку параметров окружения и подключение к базовым сервисам, так же и к базе данных);
	\item \textbf{run} предназначен для управления запуском и жизненным циклом серверного приложения, включая старт, остановку и мониторинг его работы.
\end{itemize}


\paragraph{Вспомогательные функции.}
Для поддержки работы основных классов выделен отдельный модуль \texttt{utils}, содержащий функции:
\begin{itemize}
	\item \textbf{json\_response} формирование JSON-ответов для клиентской части;
	\item \textbf{forbidden\_response} генерация ответа 403 Forbidden при отсутствии прав доступа;
	\item \textbf{get\_mime} определение MIME-типа для корректной передачи файлов;
	\item \textbf{check\_group\_permissions} проверка прав доступа пользователя в группе по заданным параметрам;
	\item \textbf{hash\_password} и \textbf{check\_password} функции для безопасного хеширования паролей и их верификации.
\end{itemize}

